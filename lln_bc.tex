\documentclass[12pt,a4paper]{article}  % Use this line if this document will be released
%\documentclass[12pt,a4paper,draft]{article}  % Use this line if this document is a draft
\usepackage{ifdraft}


%% Bibliography
\usepackage{etoolbox}
\newcommand{\bibfile}{\jobname.bib}  % Name of the BibTeX file.
% ref.bib should be a symbolic link to the universal BibTeX file, which should be a local copy of
% https://github.com/equipez/bibliographie/blob/main/ref.bib
% When the paper is finished, run `getbib` in the current directory to get the BibTeX file
% containing only the cited references. The name will be xyz.bib if this TeX file is xyz.tex.
\newcommand{\universalbib}{ref.bib}
\ifdraft{\IfFileExists{\universalbib}{\renewcommand{\bibfile}{\universalbib}}{}}{}
% 20221018: \iscite does not work as intended.
% The counter `cite' is used to count the number of citations.
\newcounter{cite}
\pretocmd{\cite}{\stepcounter{cite}}{}{}


%% Geometry
\voffset=-1.5cm \hoffset=-1.4cm \textwidth=16cm \textheight=22.0cm  % Luis' setting
%\usepackage[a4paper, textwidth=16.0cm, textheight=22.0cm]{geometry}
\renewcommand{\baselinestretch}{1.2}


%% Basic packages
\usepackage{amsmath,amsthm,amssymb,amsfonts}
\usepackage{mathtools}  % Provides \coloneqq
\usepackage{empheq}
\usepackage{xcolor}
\usepackage[bbgreekl]{mathbbol}
\DeclareSymbolFontAlphabet{\mathbbm}{bbold}
\DeclareSymbolFontAlphabet{\mathbb}{AMSb}
\usepackage{bbm}
\usepackage{upgreek}
\usepackage{accents}
\usepackage{xspace}
\usepackage{rotating}
\usepackage{multirow,booktabs}
\usepackage[en-US]{datetime2}


%% Format of the table of content
\usepackage[normalem]{ulem}
\usepackage[toc,page]{appendix}
\renewcommand{\appendixpagename}{\Large{Appendix}}
\renewcommand{\appendixname}{Appendix}
\renewcommand{\appendixtocname}{Appendix}
%\usepackage{sectsty}
\setcounter{tocdepth}{2}


%% Section title style
\usepackage{sectsty}
\sectionfont{\large}
\subsectionfont{\large}


%% Some colors
\definecolor{darkblue}{rgb}{0,0.1,0.5}
\definecolor{darkgreen}{rgb}{0,0.5,0.1}
\definecolor{darkyellow}{rgb}{0.65,0.65,0.01}


%% Todo notes
\ifdraft{
    \setlength{\marginparwidth}{2.42cm}
    \usepackage[tickmarkheight=3pt,textsize=small,backgroundcolor=blue!16,linecolor=purple,bordercolor=purple]{todonotes}
}{
    \newcommand{\todo}[1]{}
    \newcommand{\listoftodos}{}
}


%% Graph, tikz and pgf
%\usepackage{subfigure}
\setlength{\unitlength}{1mm}
% The \unitlength command is a Length command. It defines the units used in the Picture Environment.
\usepackage{graphicx}
%\usepackage{tikz,tikzscale,pgf,pgfarrows,pgfnodes,filecontents,tikz-cd}
\usepackage{tikz,tikzscale,pgf}
\usetikzlibrary{arrows,arrows.meta,patterns,positioning,decorations.markings,shapes}
\usepackage{pgfplots}
\usepackage{pgfplotstable}
\usepackage[justification=centering]{caption}
\usepgfplotslibrary{fillbetween}
\pgfplotsset{compat=1.11}


%% Turn off some unharmful warnings in draft mode
%% N.B.: DO NOT use `silence` together with `hyperref`. They will cause an infinite loop.
\ifdraft{
    \usepackage{silence}
    \WarningFilter{xcolor}{Incompatible color definition on}
    \WarningFilter{hyperref}{Draft mode on}
    \WarningFilter{refcheck}{Unused label}
    \WarningFilter{microtype}{`draft' option active}
    \WarningFilter{latex}{Writing or overwriting file} % Mute the warning about 'writing/overwriting file'
    \WarningFilter{latex}{Writing file} % Mute the warning about 'writing/overwriting file'
    \WarningFilter{latex}{Tab has} % Mute the warning about 'Tab has been converted to Blank Space'
    \WarningFilter{latex}{Marginpar on page} % Mute the warning about 'Marginpar on page xx moved'
    \WarningFilter{latex}{author given} % Mute the warning about 'No \author given'
}{}


%% Hyperref, url, and email
%% N.B.: DO NOT use `silence` together with `hyperref`. They will cause an infinite loop.
\ifdraft{\usepackage{refcheck}\newcommand{\url}{\texttt}}{
    \usepackage{hyperref}
    \hypersetup{colorlinks, linkcolor=darkblue, anchorcolor=darkblue, citecolor=darkblue, urlcolor=darkblue}
    \usepackage{url}
} % Check unused labels
\newcommand{\email}{\texttt}


%% Enumerate and itemize
\usepackage{eqlist}
\usepackage{enumitem}
\setlist[itemize]{leftmargin=*}
\setlist[enumerate]{leftmargin=*,label=\normalfont{(\alph*)}}


%% Algorithm environment
\usepackage[section]{algorithm}
\usepackage{algpseudocode,algorithmicx}
\newcommand{\INPUT}{\textbf{Input}}
\newcommand{\FOR}{\textbf{For}~}
\algrenewcommand\algorithmicrequire{\textbf{Input:}}
\algrenewcommand\algorithmicensure{\textbf{Output:}}
\algrenewcommand\alglinenumber[1]{\normalsize #1.}
\newcommand*\Let[2]{\State #1 $=$ #2}


%% Theorem-like environments
\newtheorem{theorem}{Theorem}[section]
\newtheorem{conjecture}{Conjecture}[section]
\newtheorem{corollary}{Corollary}[section]
\newtheorem{exercise}{Exercise}[section]
\newtheorem{lemma}{Lemma}[section]
\newtheorem{problem}{Problem}[section]
\newtheorem{proposition}{Proposition}[section]
\newtheorem{assumption}{Assumption}[section]
\newtheorem{example}{Example}[section]
\newtheorem{question}{Question}[section]
% Change theoremstyle to ``definition'', which uses textnormal for the text.
\theoremstyle{definition}
\newtheorem{definition}{Definition}[section]
\newtheorem{remark}{Remark}[section]
% proof
\usepackage{xpatch}
\xpatchcmd{\proof}{\itshape}{\normalfont\proofnamefont}{}{}
\newcommand{\proofnamefont}{\bfseries}

%% Equation numbering
\numberwithin{equation}{section}


%% Fine tuning
\usepackage{microtype}
\usepackage[nobottomtitles*]{titlesec} % No section title at the bottom of pages
% Prevent footnote from running to the next page
\interfootnotelinepenalty=10000
% No line break in inline math
\interdisplaylinepenalty=10000
\relpenalty=10000
\binoppenalty=10000
% No widow or orphan lines
\clubpenalty=10000
\widowpenalty=10000
\displaywidowpenalty=10000


% Use @ to put 1 math unit (mu) in math
% See https://nhigham.com/2013/01/07/fine-tuning-spacing-in-latex-equations/
% and also TeXbook p. 155.
\mathcode`@="8000{\catcode`\@=\active\gdef@{\mkern1mu}}


%% Operators, commands
\usepackage{relsize}
\usepackage{nccmath}
%\DeclareMathOperator*{\mcap}{\,\medmath{\bigcap}\,}
%\DeclareMathOperator*{\mcup}{\,\medmath{\bigcup}\,}
\DeclareMathOperator*{\mcap}{\,\mathsmaller{\bigcap}\,}
\DeclareMathOperator*{\mcup}{\,\mathsmaller{\bigcup}\,}
%\renewcommand{\cap}{\mcap}
%\renewcommand{\cup}{\mcup}

\newcommand{\ceil}[1]{ {\lceil{#1}\rceil} }
\newcommand{\floor}[1]{ {\lfloor{#1}\rfloor} }

\DeclareMathOperator{\tr}{tr}
\DeclareMathOperator{\sort}{sort}
\DeclareMathOperator*{\Argmax}{Argmax}
\DeclareMathOperator*{\Argmin}{Argmin}
\DeclareMathOperator*{\Arglocmin}{Arglocmin}
\DeclareMathOperator*{\argmax}{argmax}
\DeclareMathOperator*{\argmin}{argmin}
\DeclareMathOperator*{\diag}{diag}
\DeclareMathOperator*{\Diag}{Diag}
\DeclareMathOperator{\Span}{span}
\DeclareMathOperator{\med}{med}
\DeclareMathOperator{\essinf}{essinf}
\DeclareMathOperator{\cl}{cl}
\DeclareMathOperator{\vol}{vol}
\DeclareMathOperator{\sign}{sign}
\DeclareMathOperator{\rank}{rank}
\DeclareMathOperator{\range}{range}
\DeclareMathOperator{\card}{card}
\DeclareMathOperator{\diam}{diam}
\DeclareMathOperator{\dist}{dist}
\newcommand{\disth}{{\operatorname{\updelta_{\sss{H}}}}}
\newcommand{\ind}{\mathbbm{1}}
%\newcommand*{\defeq}{\stackrel{\mbox{\normalfont\tiny{\textnormal{def}}}}{=}}
\newcommand\defeq{\mathrel{\overset{\makebox[0pt]{\mbox{\normalfont\tiny\sffamily def}}}{=}}}

\newcommand{\RR}{\mathbb{R}}
\newcommand{\BB}{\mathcal{B}}
\renewcommand{\SS}{\mathbb{S}}
\newcommand{\TT}{\mathcal{T}}
\newcommand{\ZZ}{\mathbb{Z}}
\newcommand{\NN}{\mathbb{N}}
\newcommand{\FF}{\mathcal{F}}
\newcommand{\CC}{\mathbb{C}}
\newcommand{\XX}{\mathcal{X}}
\newcommand{\sset}{\mathcal{S}}
\newcommand{\pen}{h}
\newcommand{\penpar}{\mu}
\newcommand{\res}{\rho}
\newcommand{\col}{r}
\newcommand{\ofd}{\mathcal{F}}
\newcommand{\stf}[1]{\mathbb{S}^{#1}}
\newcommand{\sss}[1]{{\scriptscriptstyle{#1}}}
\newcommand{\sK}{{\scriptscriptstyle{K}}}
\newcommand{\sT}{{\scriptscriptstyle{T}}}
\newcommand{\io}{{\text{i.o.}}\xspace}
\newcommand{\as}{{\text{a.s.}}\xspace}
\newcommand{\comp}{{\textnormal{c}}}
\newcommand{\fro}{{\scriptstyle{\textnormal{F}}}}
\newcommand{\trs}{{\scriptstyle{\mathsf{T}}}}
\newcommand{\hmt}{{\scriptstyle{{\mathsf{H}}}}}
\newcommand{\pin}{{\scriptstyle{{\mathsf{+}}}}}
\newcommand{\inv}{{-1}}
\newcommand{\adj}{*}
\newcommand{\ones}{\mathbf{1}}

\newcommand{\cs}{\text{c}}
\newcommand{\hp}{\circ}
\newcommand{\cc}{\sss{\textnormal{C}}}
\newcommand{\dec}{\sss{\textnormal{D}}}
\newcommand{\cauchy}{\sss{\textnormal{C}}}
\newcommand{\scauchy}{\sss{\textnormal{S}}}
\newcommand{\crit}{\textnormal{crit}}
\newcommand{\rsg}{\hat{\partial}}
\newcommand{\gsg}{\partial}
\newcommand{\dom}{\textnormal{dom}}
\newcommand{\tf}{{\textnormal{f}}}
\newcommand{\tg}{{\textnormal{g}}}
\newcommand{\ts}{{\textnormal{s}}}
\newcommand{\st}{\textnormal{s.t.}}
\newcommand{\etc}{{etc.}}
\newcommand{\ie}{{i.e.}}
\newcommand{\eg}{{e.g.}}
\newcommand{\etal}{{et al.}}
\newcommand{\iid}{{\text{i.i.d.}}\xspace}

\newcommand{\me}{\mathrm{e}}
\newcommand{\md}{\mathrm{d}}
\newcommand{\mi}{\mathrm{i}}
\newcommand{\lev}{\mathrm{lev}}
\newcommand{\bA}{\mathbf{A}}
\newcommand{\bx}{\mathbf{u}}
%\newcommand{\bb}{\mathbf{f}}
\newcommand{\bb}{\mathbf{r}}
\newcommand{\nov}{n_{\textnormal{o}}}
\xspaceaddexceptions{]\}}
% tex.stackexchange.com/questions/15252/why-does-xspace-behave-differently-for-parenthesis-vs-braces-brackets
\newcommand{\MATLAB}{\textsc{Matlab}\xspace}
\newcommand{\octave}{\mbox{GNU Octave}\xspace}
\newcommand{\prblm}{\texttt}
\DeclareMathAlphabet{\mathsfit}{T1}{\sfdefault}{\mddefault}{\sldefault}
\SetMathAlphabet{\mathsfit}{bold}{T1}{\sfdefault}{\bfdefault}{\sldefault}
\newcommand{\prbb}{\mathsfit{p}}
\newcommand{\pp}{\mathsf{p}}
\newcommand{\qq}{\mathsf{q}}
\newcommand{\ttt}{\mathsfit{t}}
\newcommand{\tol}{\varepsilon}
\newcommand{\bt}{\mathbf{t}}
\newcommand{\br}{\mathbf{r}}
\newcommand{\dd}{\mathbf{d}}
\newcommand{\ii}{\mathbf{i}}
\newcommand{\jj}{\mathbf{j}}
\newcommand{\xx}{\mathbf{x}}
\renewcommand{\pp}{\mathbf{p}}
\renewcommand{\ggg}{\mathbf{g}}
\newcommand{\GG}{\mathbf{G}}
\DeclareMathOperator{\var}{Var}
\DeclareMathOperator{\cov}{Cov}
\DeclareMathOperator{\expc}{\mathbb{E}}
\renewcommand{\Pr}{\mathbb{P}}
\newcommand{\lb}{\underline}
\newcommand{\ub}{\overline}

% mathlcal font
\DeclareFontFamily{U}{dutchcal}{\skewchar\font=45 }
\DeclareFontShape{U}{dutchcal}{m}{n}{<-> s*[1.0] dutchcal-r}{}
\DeclareFontShape{U}{dutchcal}{b}{n}{<-> s*[1.0] dutchcal-b}{}
\DeclareMathAlphabet{\mathlcal}{U}{dutchcal}{m}{n}
\SetMathAlphabet{\mathlcal}{bold}{U}{dutchcal}{b}{n}

% mathscr font (supporting lowercase letters)
%\usepackage[scr=dutchcal]{mathalfa}
%\usepackage[scr=esstix]{mathalfa}
%\usepackage[scr=boondox]{mathalfa}
%\usepackage[scr=boondoxo]{mathalfa}
\usepackage[scr=boondoxupr]{mathalfa}
%\newcommand{\model}{\mathscr{h}}
\newcommand{\model}{\tilde{f}}
\newcommand{\rmod}{F}

\newcommand{\Set}[1]{\mathcal{#1}}
\DeclareMathAlphabet{\mathpzc}{OT1}{pzc}{m}{it} % The mathpzc font
\newcommand{\slv}{\mathpzc}
% mathpzc looks great, but it stops working on 19 Feb 2020 for no reason.
%\newcommand{\slv}{\mathscr}
\newcommand{\software}{\texttt}
\DeclareMathOperator{\eff}{\mathsf{e}\;\!}
\DeclareMathOperator{\Eff}{\mathsf{E}\;\!}
\newcommand{\out}{{\text{out}}}


%% Commands for revision
\newcommand{\red}[1]{\textcolor{red}{#1}}
\newcommand{\blue}[1]{\textcolor{blue}{#1}}
\newcommand{\green}[1]{\textcolor{darkgreen}{#1}}
\newcommand{\TYPO}[1]{{\color{orange}{#1}}}
\newcommand{\MISTAKE}[1]{{\color{violet}{#1}}}
\newcommand{\REPHRASE}[1]{{\color{darkgreen}{#1}}}
\newcommand{\REVISE}[1]{{\color{blue}{#1}}}
\newcommand{\REVISEred}[1]{{\color{red}{#1}}}
\newcommand{\COMMENT}{\todo}  % Needs the todonotes package
%\newcommand{\COMMENT}[1]{\textcolor{brown}{{\small{(comment: #1)}}}}  % This puts comments inline

% Use the following if revision is finished
%\newcommand{\TYPO}{}
%\newcommand{\MISTAKE}{}
%\newcommand{\REPHRASE}{}
%\newcommand{\REVISE}{}
%\newcommand{\REVISEred}{}
%\newcommand{\COMMENT}[1]{}  % Input ignored.


%%%%%%%%%%%%%%%%%%%%%%%%%%%%%%%%%%%%%%%%%%%%%%%%%%%%%%%%%%%%%%%%%%%%%%%%%%%%%%%%%%%%%%%%%%%%%%%%%%%%
\title{Notes on the Law of Large Numbers and the Borel-Cantelli Lemmas}

\date{\DTMnow}

\author{Zaikun Zhang
    \thanks{Email: zaikunzhang@gmail.com.}
    %\and
    %Author2
    %\thanks{Information2}
}


\begin{document}

\maketitle

%\begin{abstract}
%\end{abstract}

%\textbf{Keywords}: Keyword1, Keyword2
%%%%%%%%%%%%%%%%%%%%%%%%%%%%%%%%%%%%%%%%%%%%%%%%%%%%%%%%%%%%%%%%%%%%%%%%%%%%%%%%%%%%%%%%%%%%%%%%%%%%

For a sequence of random variables~$\{X_k\}$, we will say it is nonnegative if~$X_k \ge 0$ \as for
each~$k$, and uniformly bounded if there exists a constant~$M$ such that~$|X_k| \le M$ \as for
each~$k$. In addition, we define~$\bar{X}_k = k^{-1}\sum_{\ell=1}^k X_\ell$ for each~$k \ge 1$.


\section{The strong law of large numbers}
\label{sec:slln}

\subsection{The \iid case}
\label{sec:sllniid}

\begin{theorem}
    \label{th:sllniid}
    Let~$\{X_k\}$ be a sequence of \iid random variables.
    \begin{enumerate}
        \item $\{\bar{X}_k\}$ converges \as to a finite random variable if and only if~$\expc(|X_1|) < \infty$.
            When it converges \as, the limit is~$\expc(X_1)$.
        \item If~$\{X_k\}$ is nonnegative, then~$\{\bar{X}_k\}$ converges to~$\expc(X_1)$ \as
    \end{enumerate}
\end{theorem}

\begin{proof}
    (a) See \cite[Theorem~5.23]{Kallenberg_2021}.

    (b) If~$\expc(X_1) < \infty$, then the convergence holds according to~(a). Otherwise, for
    each integer $n \ge 1$, we have
    \[
        \liminf_{k\to \infty}\bar{X}_k \;\ge\;
        \liminf_{k\to \infty}\frac{1}{k}\sum_{\ell=1}^k \min\{X_\ell, n\}
        \;=\; \expc(\min\{X_1, n\}).
    \]
    Letting~$n\to \infty$, we have~$\liminf_{k}\bar{X}_k = \infty$ \as, because~$\expc(\min\{X_1,
    n\}) \to \expc(X_1) = \infty$ according to Levi's monotone convergence theorem.
    Hence~$\bar{X}_k\to \infty =\expc(X_1)$ \as
\end{proof}

\begin{remark}
    \label{rem:pair}
    For the ``if'' part of Theorem~\ref{th:sllniid}, see
    also~\cite[Theorem~1]{Etemadi_1981} and~\cite[Theorem~2.4.1]{Durrett_2019}, which show that
    the mutual independence assumption can be weakened to pairwise independence.
\end{remark}

\begin{remark}
    \label{rem:kolmogorov01}
    Suppose that~$\{X_k\}$ is a sequence of mutually independent random variables.
    If~$\{\bar{X}_k\}$ converges \as, then the limit is a constant \as according to Kolmogorov's
    zero-one law.
    For the same reason, if~$\{X_k\}$ converges \as, then its limit is a constant \as.
\end{remark}

Item~(b) of Theorem~\ref{th:sllniid} can be generalized to the following, with a very similar proof.

\begin{theorem}
    \label{th:sllniid2}
    Let~$\{X_k\}$ be a sequence of~\iid random variables. If~$\expc(X_1^+) < \infty$
    or~$\expc(X_1^-) < \infty$, then~$\{\bar{X}_k\}$ converges to~$\expc(X_1)$ \as
\end{theorem}

\begin{proof}
    If~$\expc(X_1^+)<\infty$ and~$\expc(X_1^-) < \infty$, then~$\expc(|X_1|) < \infty$. Thus the
    convergence holds. If~$\expc(X_1^+) = \infty$ and~$\expc(X_1^-) < \infty$,
    then~$\expc(X_1) = \infty$ and the convergence can be established by considering the truncated
    sequence $\{\min\{X_k, n\}\}$ for each integer~$n\ge 1$.
    The case with~$\expc(X_1^+) < \infty$ and~$\expc(X_1^-)$ is similar to the last case.
\end{proof}


\subsection{The martingale-difference case}
\label{sec:md}

\begin{theorem}[Martingale convergence theorem~\mbox{\cite[Theorem 4.2.11]{Durrett_2019}}]
    \label{th:ml1}
    If~$\{X_k\}$ is a submartingale with~$\sup_k\expc(X_k^+)< \infty$, then~$X_k\to X$ \as for
    some random variable~$X$ with~$\expc(|X|)<\infty$.
\end{theorem}

\begin{corollary}[~\mbox{\cite[Theorem 4.2.12]{Durrett_2019}}]
    \label{coro:ml1}
    If~$\{X_k\}$ is a nonnegative supermartingale, then $X_k\to X$ \as for some random variable~$X$
    with~$\expc(X) \le \expc(X_1)$.
\end{corollary}

By Theorem~\ref{th:ml1}, if~$\{X_k\}$ is an~$L^1$-bounded martingale,
then~$X_k\to X$~\as for some random variable~$X\in L^1$. However, the convergence may not hold
in~$L^1$. See~\cite[Example~4.2.13]{Durrett_2019}. In contrast, the following theorem ensures
the convergence in~$L^p$ for~$p>1$.

\begin{theorem}[Martingale ~$L^p$ convergence theorem~\mbox{\cite[Theorem~4.4.6]{Durrett_2019}}] \label{th:mlp}
    If~$p> 1$ and~$\{X_k\}$ is a martingale with~$\sup_{k\ge 1} \|X_k\|_p< \infty$,
    then there exists a random variable~$X\in L^p$ such that~$X_k\to X$ \as and in~$L^p$.
\end{theorem}

\begin{lemma}[Kronecker's Lemma]
    \label{lem:kronecker}
    Let~$\{a_k\}$ be a sequence of nonnegative numbers and~$\{b_k\}$ be a non-decreasing sequence
    with~$b_k\to \infty$.
    \begin{enumerate}
        \item If~$\sum_{k=1}^\infty b_k^{-1}a_k < \infty$, then~$b_k^{-1}\sum_{\ell=1}^k a_\ell\to 0$.
        \item If~$\sum_{k=1}^\infty a_k < \infty$, then~$b_k^{-1}\sum_{\ell=1}^k b_\ell a_\ell\to 0$.
    \end{enumerate}
\end{lemma}

\begin{remark}
    \label{rem:kronecker}
    In item~(a) of Lemma~\ref{lem:kronecker}, we trivially have~$b_k^{-1} a_k\to 0$,
    and Kronecker's Lemma strengthens it to~$b_k^{-1}\sum_{\ell=1}^k a_\ell\to 0$.
    In item~(b), we trivially have~$b_k^{-1} \sum_{\ell=1}^k a_\ell\to 0$,
    and Kronecker's Lemma strengthens it to~$b_k^{-1}\sum_{\ell=1}^k b_\ell a_\ell\to 0$.
\end{remark}

\begin{definition}
    \label{def:md}
    Let $\{X_k\}$ be a sequence of random variables and~$\{\FF_k\}$ be a filtration.
    Then~$\{X_k\}$ is called a martingale difference sequence
    adapted to~$\{\FF_k\}$ if there exists a martingale~$\{S_k\}_{k\ge 0}$ adapted to~$\{\FF_k\}$
    such that~$X_k = S_k - S_{k-1}$ for each~$k\ge 1$, or, equivalently,
    if~$\mathbb{E}(|X_k|) < \infty$ and $\mathbb{E}(X_k \mathrel{|} \FF_{k-1}) = 0$ \as for each $k \ge 1$.
\end{definition}

\begin{lemma}[Orthogonality]
    \label{lem:orth}
    If~$\{X_k\}$ is a martingale difference sequence, then
    $\expc(X_iX_j) = 0$ and~$\cov(X_i, X_j) = 0$ for any distinct~$i$ and~$j$.
\end{lemma}

\begin{proof}
    For any~$i > j \ge 1$, we have
    \[
        \expc(X_iX_j) \;=\;  \expc(X_j\expc(X_i\mathrel{|}\FF_{i-1})) \;=\; 0.
    \]
    In addition, $\expc(X_i) = \expc(\expc(X_i\mathrel{|}\FF_{i-1})) = 0$ and
    similarly~$\expc(X_j) = 0$. Hence~$\cov(X_i, X_j) = 0$.
\end{proof}

\begin{theorem}[Chow~\cite{Chow_1967}]
    \label{th:sllnmd1}
    Let~$\{X_k\}$ be a martingale difference sequence. If there exists a constant~$p\ge 1$ such that
   \[
       \sum_{k=1}^\infty k^{-p-1} \expc(|X_k|^{2p}) < \infty,
   \]
   then~$\bar{X}_k \to 0$ \as
\end{theorem}

\begin{remark}
    \label{rem:chow}
    Chow's theorem is a consequence of \cite[Theorem~2]{Chow_1960} and
    \cite[Theorem~9]{Burkholder_1966}. Chow~\cite{Chow_1967} mentions these references without
    specifying the theorems. See also
    \begin{center}
    \url{https://math.stackexchange.com/questions/4965300/}.
    \end{center}
\end{remark}

\begin{theorem}[\mbox{\cite[Theorem~2.18]{Hall_Heyde_1980}}]
    \label{th:sllnmd2}
    If~$1\le p\le 2$ and~$\{X_k\}$ is a martingale difference sequence adapted to a filtration~$\{\FF_k\}$,
    then~$\bar{X}_k \to 0$ \as on the set
   \[
       \left\{ \sum_{k=1}^\infty k^{-p} \expc(|X_k|^p
       \mathrel{|} \FF_{k-1}) < \infty \right\}.
   \]
   In particular, if~$\sum_{k=1}^\infty k^{-p} \expc(|X_k|^p) < \infty$, then~$\bar{X}_k \to 0$ \as
\end{theorem}

Theorem~\ref{th:sllnmd2} follows from Kronecker's Lemma and the theorem below due to
Chow~\cite{Chow_1965}. It generalizes the Kolmogorov's two-series theorem
(Theorem~\ref{th:kolmogorov2}) to martingale difference sequences.

\begin{theorem}[\mbox{\cite[Corollary~5]{Chow_1965}}]
    \label{th:kolmogorov2c}
    If~$1\le p\le 2$ and $\{X_k\}$ is a martingale difference sequence adapted to a filtration~$\{\FF_k\}$,
    then~$\sum_{k=1}^\infty X_k$ converges \as on the set
    \[
    \left\{\sum_{k=1}^\infty \expc(|X_k|^p\mathrel{|}\FF_{k-1}) < \infty\right\}.
    \]
    In particular, if~$\sum_{k=1}^\infty \expc(|X_k|^p) < \infty$, then~$\sum_{k=1}^\infty X_k$
    converges \as
\end{theorem}

Theorem~\ref{th:kolmogorov2c} in turn is a consequence of the following lemma, which is a conditional
version of Kolmogorov's three-series theorem (Theorem~\ref{th:kolmogorov3}). See~\cite[pages~33--36]{Hall_Heyde_1980}
for more details.

\begin{theorem}[\mbox{\cite[Theorem~2.16]{Hall_Heyde_1980}}]
    \label{th:kolmogorov3c}
    Let~$\{X_k\}$ be a sequence of random variables adapted to a filtration~$\{\FF_k\}$, $c$ be a
    positive constant, and~$Y_k = X_k\ind(X_k \le c)$ for each~$k\ge 1$. Then~$\sum_{k=1}^\infty X_k$
    converges \as on the set
    \[
        \left\{
            \sum_{k=1}^\infty \ind(X_k > c \mathrel{|} \FF_{k-1}) < \infty,
            \sum_{k=1}^\infty \expc(Y_k \mathrel{|} \FF_{k-1})~\text{converges},
            \sum_{k=1}^\infty \var(Y_k \mathrel{|} \FF_{k-1}) < \infty
        \right\}.
    \]
\end{theorem}

\begin{corollary}
    \label{coro:mdl2}
    Let~$\{X_k\}$ be a martingale difference sequence. If~$\sum_{k=1}^\infty k^{-2}\var(X_k) < \infty$,
    then~$\bar{X}_k \to 0$ \as
\end{corollary}

\begin{proof}
    This is the special case of Theorem~\ref{th:sllnmd1} with~$p=1$
    or that of Theorem~\ref{th:sllnmd2} with~$p=2$.
    We now provide an alternative proof using the martingale $L^p$ convergence theorem and
    Kronecker's Lemma.
    Define~$Y_k \;=\;  \sum_{\ell=1}^k \ell^{-1}X_\ell$ for each~$k\ge 1$. Then~$\{Y_k\}$ is
    a martingale. In addition, due to Lemma~\ref{lem:orth}, we have
    \[
        \expc(Y_k^2) \;=\; \sum_{\ell=1}^k \ell^{-2}\expc(X_\ell^2) \;\le\; \sum_{\ell=1}^\infty \ell^{-2}
        \expc(X_\ell^2) \;<\; \infty.
    \]
    Thus~$\{Y_k\}$ is bounded in $L^2$ and hence converges \as
    Invoking Kronecker's Lemma, we have~$\bar{X}_k \to 0$ \as
\end{proof}


\subsection{The independent case with moment conditions}
\label{sec:sllnindep}

\begin{lemma}
    \label{lem:expc}
    Let~$X$ be a random and and~$p \ge 1$ be a constant.
    \begin{enumerate}
        \item $\expc(|X|) \le \expc(|X|^p)^{\frac{1}{p}}$.
        \item If~$\expc(|X|) < \infty$, then~$\expc(|X - \expc(X)|^p)\le 2^p \expc(|X|^p)$.
    \end{enumerate}
\end{lemma}

\begin{proof}
    (a) Jensen's inequality (or H\"older's inequality).

    (b) By the triangle inequality, we have~$\|X - \expc(X)\|_p \le \|X\|_p + |\expc(X)| \le 2\|X\|_p$.
\end{proof}

\begin{theorem}
    \label{th:sllnindep}
    Let~$\{X_k\}$ be a sequence of independent random variables. If
    either
   \begin{equation}
    \label{eq:sum1}
    \sum_{k=1}^\infty k^{-p-1}\expc(|X_k|^{2p}) \;<\; \infty
   \end{equation}
   for a constant~$p\ge 1$, or
   \begin{equation}
    \label{eq:sum2}
    \sum_{k=1}^\infty k^{-p}\expc(|X_k|^{p}) \;<\; \infty
   \end{equation}
   for a constant~$p\in[1, 2]$, then~$\bar{X}_k - \expc(\bar{X}_k)\to 0$ \as
\end{theorem}

\begin{proof}
    According to Lemma~\ref{lem:expc}, we have~$\expc(|X_k|) < \infty$ for each~$k\ge 1$ in both cases.
    Define~$Y_k = X_k - \expc(X_k)$ for each~$k\ge 1$. Then~$\{Y_k\}$ is a martingale difference
    sequence. In addition,
    \[
        \expc(|Y_k|^{p}) \;\le\; 2^{p} \expc(|X_k|^p)
    \]
    for all~$k\ge 1$ and~$p\ge 1$. Hence~\eqref{eq:sum1} implies
   \begin{equation}
    \label{eq:sum1}
    \sum_{k=1}^\infty k^{-p-1}\expc(|Y_k|^{2p}) \;<\; \infty,
   \end{equation}
    and~\eqref{eq:sum2} implies
   \begin{equation}
    \label{eq:sum2}
    \sum_{k=1}^\infty k^{-p}\expc(|Y_k|^{p}) \;<\; \infty.
   \end{equation}
   In both cases, $\bar{Y}_k \to 0$ \as by Theorems~\ref{th:sllnmd1} and~\ref{th:sllnmd2}.
   Thus~$\bar{X}_k - \expc(\bar{X}_k) \to 0$ \as
\end{proof}

\begin{theorem}[Kolmogorov's two-series theorem~\mbox{\cite[Theorems~2.5.6]{Durrett_2019}}]
    \label{th:kolmogorov2}
    Let~$\{X_k\}$ be a sequence of independent random variables.
    Then~$\sum_{k=1}^\infty X_k$ converges \as if the two series~$\sum_{k=1}^\infty \expc(X_k)$
    and~$\sum_{k=1}^\infty \var(X_k)$ both converge.
\end{theorem}

\begin{remark}
    \label{rem:kolmogorov2}
    Theorem~\ref{th:kolmogorov2} is a special case of Theorem~\ref{th:mlp} with~$p=2$.
    To see this, define~$Y_k = \sum_{\ell=1}^k [X_\ell-\expc(X_\ell)]$ for each~$k\ge 1$. Then~$\{Y_k\}$ is a
    martingale. In addition, it is bounded in~$L^2$, since
    \[
        \expc(Y_k^2) \;=\; \sum_{\ell=1}^k \var(X_\ell)
        \;\le\; \sum_{\ell=1}^\infty \var(X_\ell) \;<\; \infty.
    \]
    Thus~$\{Y_k\}$ converges \as Combining this with the convergence of~$\sum_{k=1}^\infty \expc(X_k)$,
    we obtain the almost sure convergence of~$\sum_{k=1}^\infty X_k$.
\end{remark}

\begin{theorem}[Kolmogorov's three-series theorem~\mbox{\cite[Theorems~2.5.8]{Durrett_2019}}]
    \label{th:kolmogorov3}
    Let~$\{X_k\}$ be a sequence of independent random variables, $c$ be a positive constant,
    and~$Y_k = X_k\ind(|X_k| \le c)$. Then~$\sum_{k=1}^\infty X_k$ converges \as \textbf{if and
    only if} the three series~$\sum_{k=1}^\infty \Pr(|X_k| > c)$, $\sum_{k=1}^\infty \expc(Y_k)$, and
    $\sum_{k=1}^\infty \var(Y_k)$ all converge.
\end{theorem}

The following proposition is sometimes known as ``Kolmogorov's Criterion of SSLN''.
It follows from Theorem~\ref{th:kolmogorov2} and Kronecker's Lemma
(consider the sequence~$\{\sum_{\ell=1}^k\ell^{-1}X_\ell\}$).
It is also a special case of Corollary~\ref{coro:mdl2} and Theorem~\ref{th:sllnindep}.

\begin{corollary}
    \label{coro:kolmogorov2}
    Let~$\{X_k\}$ be a sequence of independent random variables with~$\expc(X_k)=0$ for each~$k\ge 0$.
    If~$\sum_{k=1}^\infty k^{-2}\var(X_k) < \infty$, then~$\bar{X}_k\to 0$ \as
\end{corollary}

As a side note, we have the following result due to L\'evy.

\begin{theorem}[L\'evy~\mbox{\cite[Theorem 5.3.4]{Chung_2001}}]
    \label{th:levy}
    If~$\{X_k\}$ is a sequence of independent random variables
    then~$\sum_{k=1}^\infty X_k$ converges \as if and only if it converges in probability.
\end{theorem}

\begin{question}
    What if~$\sum_{k=1}^\infty X_k$ diverges to infinity in probability or \as?
\end{question}

Note that Theorem~\ref{th:levy} cannot be extended to the case with~$\{X_k\}$ being a martingale
difference sequence. In other words, for a martingale, the convergence in probability does not imply
the almost sure convergence, as illustrated by the following example.

\begin{example}[\mbox{~\cite[Example 4.2.14]{Durrett_2019}}]
    \label{exp:durret}
    Let~$\{A_k\}$ and~$\{B_k\}$ be two sequences of \iid random variables with
    \[
        \Pr(A_k = 1) = \frac{1}{k}, \quad \Pr(A_k = 0) = 1-\frac{1}{k},
        \quad \text{and} \quad
        \Pr(B_k=1) = \frac{1}{2} = \Pr(B_k = -1).
    \]
    Additionally, we require that~$\{A_k\}$ and~$\{B_k\}$ are independent of each other.
    Define~$X_1 = 0$ and
    \[
        X_{k+1} \;=\;  A_k[B_k \ind(X_k=0) + k X_k], \quad k \ge 1.
    \]
    Then~$X_k$ is independent of $\{A_\ell\}_{\ell \ge k}$, and~$\{B_\ell\}_{\ell\ge k}$  for
    each~$k\ge 1$. Thus
    \[
        \expc(X_{k+1} \mathrel{|} X_1, \ldots, X_k)
        \;=\;
        \expc(A_k) [\expc(B_k) \ind(X_k=0) + k X_k]
        \;=\; X_k, \quad k \ge 1.
    \]
    Hence~$\{X_k\}$ is a martingale. Noting that~$B_k \ind(X_k=0) + k X_k \neq 0$, we have
    \[
        \{X_{k+1} =0\} \;=\; \{A_k = 0\},
    \]
    implying that the events $\{X_k = 0\}$ \textnormal{(}$k = 1, 2, \ldots$\textnormal{)} are mutually
    independent of each other. In addition,
    \[
        \Pr(X_{k+1} \neq 0) \;=\; 1-\Pr(A_k = 0) \;=\;  \frac{1}{k}.
    \]
    Thus~$X_k\to 0$ in probability, while the second Borel-Cantelli Lemma
    \textnormal{(}Theorem~\ref{th:bc}\textnormal{)}
    ensures that~$\Pr(X_k \neq 0~\io) = 1$, preventing the almost sure convergence to zero,
    because~$\{X_k\}$ takes only integer values.
\end{example}

\begin{question}
    What if~$\{X_k\}$ is a martingale with bounded increments?
\end{question}

For a sequence of independent random variables, the convergence in probability does not imply
the  almost sure convergence either. This can be seen by considering a sequence~$\{X_k\}$ with~$\Pr(X_k = 1) = 1/k$
and~$\Pr(X_k = 0) = 1-1/k$ independently for each~$k\ge 1$ and applying the same argument as the
end of Example~\ref{exp:durret}.


\section{The Borel-Cantelli Lemmas}
\label{sec:bc}

\begin{theorem}[The Borel-Cantelli Lemmas]
    \label{th:bc}
    Let~$\{E_k\}$ be a sequence of events.
    \begin{enumerate}
        \item If~$\sum_{k=1}^\infty \Pr(E_k) < \infty$, then~$\Pr(E_k~\io) = 0$.
        \item If~$\{E_k\}$ is mutually independent and~$\sum_{k=1}^\infty \Pr(E_k)
            = \infty$, then~$\Pr(E_k~\io) = 1$.
    \end{enumerate}
\end{theorem}

\begin{remark}
    \label{rem:01}
    The Borel-Cantelli Lemmas lead to a 0-1 law: For a sequence of mutually independent
    events~$\{E_k\}$, the probability~$\Pr(\{E_k~\io\})$ is either 0 or 1, corresponding
    to~$\sum_{k=1}^\infty \Pr(E_k) < \infty$ or $\sum_{k=1}^\infty \Pr(E_k) = \infty$, respectively.
\end{remark}

\begin{lemma}
    \label{lem:sumprod}
    For a sequence~$\{x_k\} \subset [0,1)$, $\sum_{k=1}^\infty x_k = \infty$ if and only if
     $\prod_{k=1}^\infty (1-x_k) = 0$.
\end{lemma}
\begin{proof}
    If~$\sum_{k=1}^\infty x_k = \infty$, then~$\prod_{k=1}^\infty (1-x_k)\le \exp(-\sum_{k=1}^\infty
    x_k) = 0$. If~$\sum_{k=1}^\infty x_k < \infty$, then there exists an integer~$K$ such
    that, for each~$k\ge K$, we have~$x_k \le 1/2$ and hence~$1-x_k \ge \exp(-2x_k)$. Thus
    $\prod_{k=K}^\infty (1-x_k) \ge \exp(-2\sum_{k=K}^\infty x_k) > 0$, implying that~$\prod_{k=1}^\infty
    (1-x_k) > 0$.
\end{proof}

\begin{theorem}
    \label{th:inf}
    Let~$\{X_k\}$ be a sequence of nonnegative random variables.
    \begin{enumerate}
        \item If~$\sum_{k=1}^\infty\expc(X_k) < \infty$, then~$\sum_{k=1}^\infty X_k < \infty$~\as
        \item If~$\sum_{k=1}^\infty\expc(X_k)= \infty$, then~$\sum_{k=1}^\infty X_k = \infty$~\as
            provided that the random variables~$\{X_k\}$ are~\iid or mutually independent and
            uniformly bounded.
    \end{enumerate}
\end{theorem}

\begin{proof}
    (a) By the Fubini-Tonelli theorem (or Levi's monotone convergence theorem), we have
    \[
        \expc\left(\sum_{k=1}^\infty X_k\right) \;=\; \sum_{k=1}^\infty \expc(X_k) \;<\; \infty.
    \]
    Hence~$\sum_{k=1}^\infty X_k < \infty$~\as

    (b) Suppose that~$\{X_k\}$ is \iid. By the strong law of large numbers (Theorem~\ref{th:sllniid}),
    we have~$k^{-1} \sum_{\ell=1}^k X_\ell  \to \expc(X_1)$~\as, which is positive (possibly
    infinite) since~$\sum_{k=1}^\infty\expc(X_k) = \infty$. Hence~$\sum_{k=1}^\infty X_k = \infty$~\as

    Now suppose that~$\{X_k\}$ is a sequence of mutually independent and uniformly bounded random variables.
    Without loss of generality, we assume that~$X_k< 1$ \as for each~$k\ge 1$. Then~$\expc(X_k) < 1$
    for each~$k\ge 1$. According to Lemma~\ref{lem:sumprod}, $\sum_{k=1}^\infty \expc(X_k) = \infty$
    implies that~$\prod_{k=1}^\infty [1-\expc(X_k)] = 0$.
    Due to the independence of~$\{X_k\}$, we have
    \[
        \expc\left[\prod_{k=1}^\infty (1-X_k)\right]
        \;=\; \prod_{k=1}^\infty \expc(1-X_k)
        \;=\;  \prod_{k=1}^\infty [1-\expc(X_k)]
        \;=\; 0.
    \]
    Thus~$\prod_{k=1}^\infty (1-X_k) = 0$~\as since this is a nonnegative random variable,
    implying that~$\sum_{k=1}^\infty X_k = \infty$~\as according to Lemma~\ref{lem:sumprod}.
\end{proof}

\begin{remark}
    \label{rem:bc}
    The Borel-Cantelli Lemmas are special cases of Theorem~\ref{th:gbc} with~$X_k = \ind(E_k)$.
    Note that~$\{E_k~\io\} = \{\sum_{k=1}^\infty X_k = \infty\}$.
\end{remark}

The following theorem generalizes item~(b) of Theorem~\ref{th:inf} in the uniformly bounded case.

\begin{theorem}
    \label{th:gbc}
    Let~$\{X_k\}$ be a sequence of uniformly bounded random variables adapted to
    a filtration~$\{\FF_k\}$.
    Define
    \begin{equation}
        \nonumber
        %\label{eq:gbc}
        A \;=\; \left\{ \sum_{k=1}^\infty X_k =\infty \right\}
        \quad \text{and} \quad
        B \;=\;  \left\{ \sum_{k=1}^\infty \expc(X_k\mathrel{|}\FF_{k-1}) =\infty \right\}.
    \end{equation}
    \begin{enumerate}
        \item \label{it:a}If~$\{\expc(X_k\mathrel{|}\FF_{k-1})\}$ is
            nonnegative, then $\Pr(A \cap B^\comp) = 0$.
        \item If~$\{X_k\}$ is nonnegative,
            then~$\Pr((A \cap B^\comp) \cup (B\cap A^\comp)) = 0$.
    \end{enumerate}
\end{theorem}

\begin{proof}
    This proof is essentially the same as that of~\cite[Theorem~4.3.4]{Durrett_2019}.
    Define
    \[
        Y_k \;=\;  \sum_{\ell = 1}^{k} X_\ell - \sum_{\ell = 1}^{k} \expc(X_\ell\mathrel{|}
        \FF_{\ell-1}).
    \]
    Then~$\{Y_k\}$ is a martingale with respect to the filtration~$\{\FF_k\}$, and it has bounded
    increments if~$\{X_k\}$ is uniformly bounded.
    Hence~$\Pr(C\cup D) = 1$ with
    \begin{align*}
        C \;=\; & \left\{ \lim_{k\to\infty} Y_k~\text{exists and is finite} \right\}, \\
        D \;=\; & \left\{ \liminf_{k\to \infty} Y_k = -\infty~\text{and}~\limsup_{k\to \infty} Y_k = \infty \right\}.
    \end{align*}
    It is clear that
    \begin{equation}
        \label{eq:C}
        A\cap C \;=\; B\cap C.
    \end{equation}

    Suppose that~$\{\expc(X_k\mathrel{|}\FF_{k-1}\}$ is nonnegative. Then
        \[
            B^\comp \;=\;  \left\{ \sum_{k=1}^\infty \expc(X_k\mathrel{|}\FF_{k-1}) <\infty \right\}.
        \]
    Therefore, we can check that
    \begin{equation}
        \label{eq:D1}
        A \cap B^\comp \cap D \;=\; \emptyset.
    \end{equation}
    Combining~\eqref{eq:C} and~\eqref{eq:D1}, we have
    \[
        (A \cap B^\comp) \cap (C\cup D) \;=\; \emptyset,
    \]
    which implies that~$\Pr(A\cap B^\comp) = 0$ since~$\Pr(C\cap D) = 1$.

    If~$\{X_k\}$ is nonnegative,  then so is~$\{\expc(X_k\mathrel{|}\FF_{k-1})\}$,
    and hence~$\Pr(A\cap B^\comp) = 0$. In addition, we have
    \[
        A^\comp \;=\;  \left\{ \sum_{k=1}^\infty X_k <\infty \right\},
    \]
    which implies~$\Pr(B\cap A^\comp) = 0$ in a way similar to the above.
    Thus~$\Pr((A \cap B^\comp) \cup (B\cap A^\comp)) = 0$.
\end{proof}

\begin{question}
    What if~$\{X_k\}$ is a sequence of identically distributed but not necessarily independent
    variables? What if~$\{\expc(X_k\mathrel{|}\FF_{k-1})\}$ is identically distributed?
\end{question}

\begin{remark}
    \label{rem:equal}
     $\Pr((A \cap B^\comp) \cup (B\cap A^\comp)) = 0$ means that the symmetric difference between
     the events~$A$ and~$B$ is a null event, or the two events are equal except for a null event.
\end{remark}

\begin{remark}
    \label{rem:durret}
    \cite[Theorem~4.3.4]{Durrett_2019} is a special case of Theorem~\ref{th:gbc}, where~$X_k
    = \ind(E_k)$ with an event~$E_k \in \FF_k$ for each~$k\ge 1$.
    However, the conclusion of~\cite[Theorem~4.3.4]{Durrett_2019}
    is
    \begin{equation}
        \label{eq:durret}
        \{E_k~\io\} \;=\; \left\{\sum_{k=1}^\infty \Pr(E_k\mathrel{|}\FF_{k-1})
        = \infty\right\},
    \end{equation}
    which seems not completely rigorous. The correct conclusion should be that the events on the
    left and right sides of~\eqref{eq:durret} are equal except for a null event.
    In addition, in the second line of the proof of~\cite[Theorem~4.3.4]{Durrett_2019},
    the~$A$ and~$B$ seem to be reversed. Finally, the proof refers to Doob's decomposition
    theorem~(\cite[Theorem~4.3.2]{Durrett_2019}), which seems not necessary.
\end{remark}

The following proposition shows that~$\Pr(B\cap A^\comp)$ is not necessarily zero in
item~\ref{it:a} of Theorem~\ref{th:gbc}.
\begin{proposition}
    \label{prop:sum}
    Let $\{X_k\}$ be a sequence of mutually independent random variables taking either $1$ or $-1$ with
$\mathbb{E}(X_k) = Ck^{-\alpha}$ for each $k\ge 1$,
where $C$ and $\alpha$ are positive constants.
Then, almost surely,
\[
\liminf_{k\to \infty} \sum_{\ell=1}^k X_\ell \;=\;
\begin{cases}
+\infty & \text{ if } \alpha < \dfrac{1}{2}, \\[2ex]
-\infty & \text{ if } \alpha \ge \dfrac{1}{2}.
\end{cases}
\]
\end{proposition}

\begin{proof}
According to Kolmogorov's iterated law of logarithm~\cite{Kolmogoroff_1929} (see
also~\cite[Theorems~7.1--7.3]{Petrov_1995}), we have
\begin{equation}
    \label{eq:kolmogorov}
    \liminf_{k\to\infty} \frac{\sum_{\ell=1}^k [X_\ell -\expc(X_\ell)]}{\sqrt{2B_k\log\log B_k}} \; =\; -1,
\end{equation}
where~$B_k = \sum_{\ell=1}^k \var(X_\ell)$. The law is applicable because~$B_k\to \infty$
and
\[
\|X_k-\expc(X_k)\|_\infty = o(\sqrt{B_k/\log\log B_k}).
\]
Since~$B_k = \sum_{\ell=1}^k (1-1/\ell)$, it is clear that~$(B_k \log\log B_k)/(k\log\log k) \to 1$.
Hence~\eqref{eq:kolmogorov} implies
\begin{equation}
    \label{eq:sumXdE}
    \liminf_{k\to\infty} \frac{\sum_{\ell=1}^k [X_\ell -\expc(X_\ell)]}{\sqrt{2k\log\log k}} \; =\; -1.
\end{equation}
Meanwhile, since~$\expc(X_k) = Ck^{-\alpha}$, we have
\begin{equation}
    \label{eq:sumE}
    \lim_{k\to \infty} \frac{\sum_{\ell=1}^\infty \expc(X_\ell)}{\sqrt{2k\log\log k}} \;=\;
    \begin{cases}
        +\infty & \text{ if } \alpha < \dfrac{1}{2}, \\[2ex]
        0 & \text{ if } \alpha \ge \dfrac{1}{2}.
    \end{cases}
\end{equation}
Combining~\eqref{eq:sumXdE} and~\eqref{eq:sumE}, we have
\[
    \liminf_{k\to\infty} \frac{\sum_{\ell=1}^k X_\ell}{\sqrt{2k\log\log k}} \; =\;
    \begin{cases}
        +\infty & \text{ if } \alpha < \dfrac{1}{2}, \\[2ex]
        -1 & \text{ if } \alpha \ge \dfrac{1}{2},
    \end{cases}
    \]
    which implies the desired conclusion.
\end{proof}

\begin{remark}
    \label{rem:ms}
    For more discussions on Proposition~\ref{prop:sum}, including a perspective from the
    Kakutani's Dichotomy theorem~\cite{Kakutani_1948}, see
    \begin{center}
        \url{https://math.stackexchange.com/questions/4968996/}.
    \end{center}
\end{remark}



%%%%%%%%%%%%%%%%%%%%%%%%%%%%%%%%%%%%%%%%%%%%%%%%%%%%%%%%%%%%%%%%%%%%%%%%%%%%%%%%%%%%%%%%%%%%%%%%%%%%
%% References
% Include references only if there are citations.
\ifnum\value{cite}>0
    \small
    \bibliography{\bibfile}
    \bibliographystyle{plain}
\fi

%% The end
\end{document}
